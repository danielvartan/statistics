\section{Séries Temporais \cite{torres}}

	\subsection{O que é?}

        Uma série temporal consiste em um conjunto de observações de variáveis quantitativas coletadas ao longo do tempo. (...)
        
    \subsection{Componentes de Séries Temporais}
    
        - Tendência - Crescente ou decrescente
        
        - Variações cíclicas - Período longo
        
        - Variações sazonais - Período curto (1 ano)
        
        - Variações irregulares ou aleatórias
        
        $ST = T + \times \ VS + \times \ VC + \times VA$
        
        \subsubsection{Tendência (T):}
        
        \subsubsection{Variações Cíclicas (C):}
        
        \subsubsection{Variações Sazonais (S):}
        
        \subsubsection{Variações Irregulares ou Aleatórias (I):}
        
    \subsection{Processo de Análises}
    
        (...)
    
        \subsubsection{Forma aditiva:}
        
            Considera que a série temporal é uma soma dos quatro componentes:
            
            $Y = T + C + S + I$
        
        \subsubsection{Forma multiplicativa:}
        
            Considera que a série temporal é um produto dos quatro componentes:
            
            $Y = T \times C \times S \times I$
            
    \subsection{Forma Aditiva:}
    
        (...)
        
    \subsection{Formas Multiplicativa:}
    
        (...)
        
    \subsection{Técnicas de Suavização}
    
        \subsubsection{Média Móvel Simples (MMS):}
        
            Uma média móvel (MM) erá o efeito de "alisar" os dados, produzindo um movimento com menos picos e vales. (...)
            
            (...)
            
        \subsubsection{Análise da Qualidade da Previsão}
        
            (...)
            
            Desvio Médio Absoluto: 
            
            Erro Quadrático Médio (EQM)
            
        \subsubsection{Média Móvel Ponderada (MMP)}
        
            (...)
            
            