\section{Análise Fatorial \cite{torres}}

    (...)

	\subsection{O que é?}

        É o conjunto de técnicas estatísticas, que procura explicar a correlação entre as variáveis observáveis, simplificando os dados através da redução do número...
        
    \subsection{Objetivos}
    
        O objetivo principal da análise fatorial é a redução da massa de dados altamente relacionados. Entretanto, a redução não implica em admitir perda do comportamento das variáveis originais. Nesse sentido quer-se reduzir a massa de dados, mas ao mesmo tempo, preservar ao máximo as características comportamentais do conjunto de variáveis.
        
        A análise fatorial também é utilizada para gerar fatores compostos pelas variáveis originais, que podem ser utilizados em outras técnicas de análise multivariada.
        
        Como os fatores gerados em uma análise fatorial podem não apresentar correlação entre si, então podem ser utilizados para resolver o problema de variáveis multi-correlacionadas na regressão.
        
    \subsection{Dimensão da Amostra}
    
        O mínimo de respostas válidas ($N$) por variáveis ($K$) é:
        
        $
        N = 50 \text{se} K <= 5
        N = 10 * K se 5 < K <= 15
        N = 5 * K se K > 15
        $
        
        (Gráfico)
        
    \subsection{Análise Fatorial Exploratória e de Confirmação}
    
    \subsection{Exemplo:}
    
        Phil gostaria de saber se poderia simplificar seu entendimento das percepções dos dois restaurantes, reduzindo o número de variáveis para menos de 12 (tratar como amostra única). Isto é, se ele pode representar as 12 variáveis originais de percepção ($X_{1}$ até $X_{12}$) com um número menor de fatores significativos. A que resultados chegou o consultor?
        
        \subsubsection{Variáveis do banco de dados}
        
            \textbf{X1} - Comida de excelente qualidade
            
            \textbf{X2} - Um interior atraente
            
            \textbf{X3} - Porções generosas
            
            \textbf{X4} - Comida de gosto excelente
            
            \textbf{X5} - Bom valor para o dinheiro
            
            \textbf{X6} - Funcionários simpáticos
            
            \textbf{X7} - Aparência limpa e organizada
            
            \textbf{X8} - Um ambiente divertido
            
            \textbf{X9} - Grande variedade de pratos
            
            \textbf{X10} - Preços razoáveis
            
            \textbf{X11} - Funcionários gentis
            
            \textbf{X12} - Funcionários competentes
            
    \subsection{Premissas:}
    
        A análise fatorial possui algumas premissas:
        
        1a \textbf{As variáveis devem ser medidas em escala intervalar ou razão};
        
        2a As variáveis envolvidas na análise devem ser normalmente distribuídas;
        
        3a Se as variáveis envolvidas na análise estiverem medidas em escalas diferentes, as mesmas devem ser padronizadas antes de proceder a análise.
        
        \subsubsection{Quando os Dados não são Normais}
    
            Neste caso, devemos proceder da seguinte maneira: (1) usar uma transformação matemática que normalize a variável e usar a nova variável no lugar da original não normal; ou (2) alguns autores dizem que uma variável mesmo que não seja normalmente distribuída, pode ser incluída na análise fatorial, se apresentar simetria.
        
            Para verificar se as variáveis, mesmo não sendo normais, tem distribuição simétrica: calcular a assimetria e dividir (Stat/Std. Error). Se o valor estiver contido no intervalo [-1.96; 1,96] podemos supor que exista uma distribuição simétrica.
        
    \subsubsection{Próximo Passo}
        
    \subsection{Rotação de Fatores}
        
        A matriz de fatores contém os coeficientes utilizados para expressar as variáveis em termos dos fatores. Esses coeficientes, chamados de cargas fatoriais, representam as correlações entres os fatores e as variáveis. Um coeficiente com valor absoluto grande, indica que o fator e a avariável estão estritamente relacionados.
            
        Ao fazer a rotação dos fatores, seria interessante que cada variável tivesse coeficiente diferente de zero ou significativos com poucos fatores, sendo o ideal com apenas um, pois se vários fatores têm altas cargas com a mesma variável, tona difícil interpretá-los.
            
        \subsubsection{Tipo de Rotação de Fatores:}
            
            \textbf{Ortogonais}: rotação de fatores em que os eixos são mantidos em ângulos retos. A rotação ortogonal tem como resultado fatores não-correlacionados (varimax, quartimax e equamax). O mais utilizado é o varimax, que minimiza o número de variáveis com altas cargas sobre um fator, facilitando assim, a interpretação dos fatores.
                
            \textbf{Oblíquas}: rotação de fatores em que os eixos não são mantidos em ângulos retos. Deve-se utilizar rotação oblíqua quando os fatores na população tendem a ser fortemente correlacionados (direct oblimin e promax)
                
    \subsection{Tipos de Extração de Fatores:}
        
        \textbf{Componentes principais}: Geralmente usamos o padrão: \textbf{Análise de Componentes Principais}. No SPSS, bem como em outros pacotes de software de estatística, o PCA é o método padrão para a extração de análise fatorial. Ele não é especificamente um método de análise fatorial , mas é amplamente usado como um método de extração.
            
        É um procedimento estatístico multivariado, que permite transformar um conjunto de variáveis \textbf{quantitativas iniciais}, correlacionadas entre si, em outro conjunto com um menor número de variáveis não correlacionadas reduzindo a complexidade dos dados. \textbf{Não pressupõe a normalidade}, mas sua ausência ou presença de \textit{outliers} \textbf{pode} provocar distorções.
            
        As componentes principais são calculadas por ordem decrescente de importância, isto é, a primeira explica a máxima variâncias dos dados, segunda a máxima variância ainda não explicada pela primeira, e assim sucessivamente. A última componente será a que menos explica a variância total dos dados.
            
        \\
            
        \textbf{Máxima Verossimilhança}: a \textbf{normalidade é um pressuposto exigido} por esse método, isto é, ele assume que os dados provém de uma distribuição normal \textbf{multivariada}.
            
        Não é simplesmente fazer o teste de Komolgrov, é preciso calcular o teste "PK de Mardia" - que junto as variáveis em uma só e testa sua normalidade;
            
        \\
            
        \textbf{Eixo Principal de Fatoração ou Análise Fatorial Comum}: Opte pela Análise Fatorial Comum \textbf{se os seus dados são significativamente anormais}. Esse método de extração ...
        
    \subsection{KMO}
    
        Para se poder aplicar o modelo fatorial, deve haver correlação entre as variáveis. Se essas correlações forem pequenas é pouco provável que partilhem fatores comuns.
        
        O KMO é um procedimentto estatístico que permite aferir a qualidade das correlações entre as variáveis de forma a prosseguir com a análise fatorial
        
        ...
        
    \subsection{Teste de Esfericidade de Bartlett}
        
        O teste de Esfericidade de Bartlett testa a hipótese de que há correlação entre algumas variáveis.
        
        \textbf{Este teste requer que os dados provenham de uma população normal multivariada}. No entanto, este teste é muito influenciado pelo tamanhos da amostra, e eleva a rejeitar a hipótese nula em grandes amostras. Neste caso, é preferível o uso do KMO.
    
    \subsection{Extração de Fatores}
    
        O número de fatores necessários para descrever os dados, pode ser obtido através de um dos seguintes procedimentos:
        
        *K = número de variáveis.
        
        \\
        
        1. \textbf{para $K <= 30$}, usar o critério de Kaiser, pelo qual se escolhem os fatores cuja variância explicada é superior a 1 (eigenvalues $> 1$);
        
        \\
        
        2.\textbf{para $K > 30$}m usar o Scree Plot, isto é, o gráfico da variância pelo número de componente, onde os pontos no maior declive são indicativos do número apropriado de componentes a reter.
        
        \\
        
        Quando o número de casos é superior a 250 e o valor médio das comunalidades é grande ($>= 0,6$), ambos os critérios fornecem o mesmo resultado.
        
        Quando as comunalidades são pelo menos $0,6$ e o número de variáveis é inferior a 30 ou o número de observações é superior a 250, tanto o critério de Kaiser, como o Scree Plot, geram soluções confiáveis quanto ao número de fatores a reter. Essa credibilidade é aumentada quando o quociente entre o número de fatores retidos e o número de variáveis iniciais é inferior a $0,3$.
        
    \subsection{Matriz de Anti-Imagem}
    
        A matriz de anti-imagem é uma medida de adequação amostral de cada variável (MAS) para uso da análise fatorial onde pequenos valores (menores que $0,5$) na diagonal principal nos levam a considerar a eliminação da variável.
        
    \subsection{Comunalidade}
    
        Pelo quadro das comunalidades podemos observar que todas as variáveis possuem uma forte relação cpom os fatores encontrados, exceto as variáveis X7 e X10 que estão bem próximas do limite $0,6$.
        
    \subsection{Matriz de Componentes}
    
        A matriz dos componentes mostra as coeficientes ou pesos que correlacionam as variáveis aos fatores antes da rotação.
        
        Espera-se que não hajam pesos elevados, em mais de um fator, para uma mesma variável, pois isto dificultaria a interpretação. Muitas vezes a extração inicial ou anterior à rotação não fornece fatores interpretáveis.
        
        Na análise fatorial, quando há variáveis com baixos pesos, não se controla a sua influência eliminando-as e usando apenas as variáveis com pesos elevados. Cabe ao pesquisador excluí-las ou não da análise, de acordo com o fundamento teórico subjacente.
        
    \subsection{Matriz de Componentes após Rotação}
    
        A matriz dos componentes após a rotação ortogonal é útil para designar o significado dos fatores, essencialmente quando as variáveis têm pesos elevados ...
        
    \subsection{Nomeando os Fatores}
    
    ...